As a computational molecular biophysics grad student (i.e. I program and know
biochem), I disagree with this analogy. Human programming is nowhere near the
level of complexity of transcription/translation. Here is my attempt at
comparing programming with DNA:

DNA is the source code. It is transcribed(compiled) to RNA(bytecode), but what
sections and how often is determined by levels, locations, and conformations of
DNA, RNA, and proteins. RNA can inhibit or promote transcription of different
sections, and proteins can do the same, of course dependent on which pieces of
RNA/proteins and at what levels and what location. RNA is compiled to
proteins(machine code) in the same way, with the same parameters. Also, at each
level of code (source code, bytecode, machine code) there is additional
space/time dependent modification, i.e. the source code or machine code now may
differ from the source code or machine code in 2 seconds. For example, DNA has
error-prone copying, can curl up in certain sections and be unavailable for
"compiling", can have proteins/RNA bind to certain sections and stop the
compiler(polymerase) from continuing(like a breakpoint), etc.

So essentially you have a self-modifying source code which generates a ton of
sections of bytecode, all of which can modify the source code, other pieces of
bytecode, and other pieces of machine code, and a bunch of machine code pieces
generated from the all the pieces of bytecode which can modify the source-code,
some of the bytecode, and some of the machine code. The source-code for the
compiler is also included and is created/modified following the above
procedures. Oh ya, and throw in the fact that every modification/compilation is
stochastic and also based on temporal/spatial factors (imagine compilations and
modifications only happen if a compiler bumps into code, and that movement is
dictated by diffusion or assistance by a piece of byte or machine code). That is
biology.

