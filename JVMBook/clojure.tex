% %-------------------------------------- % Scala and Lift
% %--------------------------------------
\section{JVM Languages - Clojure}


Clojure is a Lisp dialect created in 2007 by Rich Hickey. It recently reached
a 1.0 release and has a large following of Java programmers and Common Lisp
developers. Since the Dec 2008 release, the Clojure library has received
18,000 downloads (based on Google Code Stats). Clojure is a dynamically
(dynamic/strong) typed language, supports lazy sequences, immutable data
structures, macros, and functions as first class objects [4]. Clojure is a
functional language just like Common Lisp is a functional language. It is a
Lisp dialect so that includes the fully parenthesized syntax. Most syntax
will include a function or macro call with arguments or no arguments enclosed
by a left and right parenthesis.

Most newcomers to a Lisp dialect may get distracted by the parentheses, the symbolic expressions. They can seem daunting if you are more familiar to a language like C++ or Java. But, this actually is one of the major benefits of Lisp. The simple syntax, functional call, args very much resemble how the compiler or parser will interpret the code. Simple is good. Simple is fast, especially to the machine. It can also benefit the developer because you aren't overburdened with a bunch of syntax to memorize.

It also really helps to have a great editor like Emacs. Emacs is built with its own Lisp dialect, Emacs Lisp. So, Clojure syntax is not too foreign to Emacs. You will need to download the Clojure Emacs Mode and you want to add Slime integration.

Here is a snippet of Clojure code. Just focus on the left parenthesis and the token adjacent to the character. The token, function or macro call and the left parenthesis.

Lisp does not normally get more complicated than the parenthesis tokens and
defining function bodies. A function/macro call and arguments. The arguments are normally separated by a variable number of spaces and may include calling another routine. That is the essence of functional programming. You have functions that return some value and can call other functions. You don't have to worry about Object creation syntax, for loop syntax, anonymous inner classes or things that you might encounter with Java. Here is some sample Java code. Look at all the tokens that part of the language.
