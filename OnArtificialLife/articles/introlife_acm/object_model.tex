%%--------------------------------------
%% Scala and Lift
%%--------------------------------------
\section{Object Model for Artificial Life Simulation}

I hope have made a case that organic life is interesting.  The machinery has
already been created but we don't have a easy blueprint to recreate such complex
life, we can only use knowledge base to model as much as we can.

Here is a model for DNA. Game of Life, Cell Grid.

2D visual simulation, cells, grid, iterations or cell game cycles, cell
entities, bacteria.

Bacteria Living bacteria cell, single celled organism.

The most basic units in the life simulation consist of
Chemical Life Elements.  This enumeration contain the element types.
The element types are closely tied to the proteins on the grid
but in reality they are synonymous with chemical life elements like Carbon, Oxygen, etc.

Each grid in the life simulation consists of a chemical element unit.
The element unit contains an element level or weight.  The elements
react with other elements.

The DNA represents the CODE in this simulation.
We will decode the DNA code for protein synthesis.

For the bacteria cell, DNA can effect:

Size, weight, energy level, color, reproduction rate, 
food consumption rate, ability to handle water, sunlight/temperatures.

All forms of algae are composed of eukaryotic cells but for our demo, we are treating 
Food/Algae as a type of non organic entity.  But the cells in our system feed on this non organic form of algae.

When constructing the object model for this DNA simulation, I wanted to focus
three key components, the DNA, mutations, cell properties or traits.  But also,
there is some validation given to the water and the sunlight.  These attributes
don't effect the system as much but are part of the environment.

Model DNA bases, Adenine, Cytosine, Guanine, Thymine.

alive, processDNA, produceProteins, onStepSimulationProcessCell,
setImmutableSystemTraits, Genes, DNATranslationParser

LivingEntityCell

\subsection{Running the Simulation} 

With this basic demo, already the system exhibits interesting properties.  We
start with a few living cells.  Over time, cells are created, and cells die but
the system stabalizes with  several dozen cells.  Once mutations start to occur,
one form of entity tends to survive and the grid changes color.
