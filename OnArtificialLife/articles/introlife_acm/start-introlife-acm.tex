%%%%%%%%%%%%%%%%%%%%%%%%%%%%%%%%%%%%%%%%%%%%%%%%%%%%%%%%%%%%%%%%%%%%%
%
% NOTE: WORK IN PROGRESS
%
% Modification to ACM sample.
% Based on:
% acmsmall-sample.tex, dated 15th July 2010
% This is a sample file for ACM small trim journals
%
% Compilation using 'acmsmall.cls' - version 1.1, Aptara Inc.
% (c) 2010 Association for Computing Machinery (ACM)
%
% Questions/Suggestions/Feedback should be addressed to => "acmtexsupport@aptaracorp.com".
% Users can also go through the FAQs available on the journal's submission webpage.
%
% Steps to compile: latex, bibtex, latex latex
%
% For tracking purposes => this is v1.1 - July 2010
%%%%%%%%%%%%%%%%%%%%%%%%%%%%%%%%%%%%%%%%%%%%%%%%%%%%%%%%%%%%%%%%%%%%%

\documentclass[prodmode,acmtecs]{acmsmall}

\usepackage{draftcopy}

\usepackage{graphicx}
\usepackage{type1cm}
\usepackage{eso-pic}
\usepackage{color}

% Package to generate and customize Algorithm as per ACM style
\usepackage[ruled]{algorithm2e}
\renewcommand{\algorithmcfname}{ALGORITHM}
\SetAlFnt{\small}
\SetAlCapFnt{\small}
\SetAlCapNameFnt{\small}
\SetAlCapHSkip{0pt}
\IncMargin{-\parindent}

% Metadata Information
\acmVolume{1}
\acmNumber{1}
\acmArticle{1}
\acmYear{2011}
\acmMonth{3}

% Document starts
\begin{document}

% Page heads
\markboth{Berlin Brown}{A Bottom-Up Approach for Artificial Life Simulations}

\makeatletter
\AddToShipoutPicture{%
            \setlength{\@tempdimb}{.5\paperwidth}%
            \setlength{\@tempdimc}{.5\paperheight}%
            \setlength{\unitlength}{1pt}%
            \put(\strip@pt\@tempdimb,\strip@pt\@tempdimc){%
        \makebox(0,0){\rotatebox{45}{\textcolor[gray]{0.75}%
        {\fontsize{6cm}{6cm}\selectfont{DRAFT}}}}%
            }%
}
\makeatother

% Title portion
\title{A Bottom-Up Approach for Artificial Life Simulations}
\author{Berlin Brown, berlin.brown@gmail.com
\affil{Berlin Research}}

\begin{abstract}
The field of artificial intelligence in computer science focuses on many
different areas of computing from computer vision to natural language
processing. These top-down approaches typically concentrate on human behavior or
other animal functions. In this article we look at a bottom-up approach to
artificial life and how emergent cell behavior can produce interesting results. 
With this bottom-up alife approach, we are not interested in solving any
particular task, but we are interested in observing the adapative nature of the
entities in our simulation. We also wanted to introduce those more familiar with
software engineering to biological systems and evolutionary theory concepts.
\end{abstract}

\category{C.2.2}{Artificial Intelligence}{Artificial Life}

\terms{Evolution, Artificial Life, ALife, Artificial Intelligence}

\keywords{Evolution, artificial life, alife, scala, java, bottom-up}

\begin{bottomstuff}
This work is supported by Berlin Research.

Author's addresses: B. Brown, Atlanta Georgia
\end{bottomstuff}

\maketitle


%%%%%%%%%%%%%%%%%%%%%%%%%%%%%%%%%%%%%%%%%%%%%%%%%%%%%%%%%%%%%%%%%%%%%%
%% General Structure Notes:
%%
%% In my articles, I will question, what is artificial intelligence?
%%        what is intelligence?  Why is human intelligence more interesting than
%%     animal intelligence?
%%
%% Outline:
%% 1. Overview
%% 2. History of AI and Computing, Turing, VonNue, etc
%%    History with ALife, Langton, GameOfLife
%%
%% Charles Darwin
%% 1937 Alan Turing
%% 1945 John von Neumann
%%      Wolfram
%% 1970 Cellular automaton, John Conway
%% 1986 Christopher Langton
%%      Marvin Minsky
%%s
%% 3. Basic biology and concepts
%%     What is DNA?
%%     What is the mitochondira?
%%     What is RNA?
%%     What is life?  Why is it interesting?
%%
%% 4. Modeling the biology (artificial life, etc), biology computing, alife
%%    Overview of the demo application, object model
%%    DNA    
%%    More detail and code.
%%    Analyzing results
%% 7. Java and Scala Swing / 
%% 9. Summary and Future Direction
%%%%%%%%%%%%%%%%%%%%%%%%%%%%%%%%%%%%%%%%%%%%%%%%%%%%%%%%%%%%%%%%%%%%%%

\tableofcontents

% Basic overview and introduction
%%--------------------------------------
%% Scala and Lift
%%--------------------------------------
\section{Standard Java Libraries}

Scala is a JVM language.

Our web-application would not be complete without a clear approach 
for persisting the link data. So we have used the Hibernate ORM 
(object relational mapping) library do the backend persistance work for us. 
It is not really necessary to use Hibernate for such a simple 
application, but as your enterprise application grows, 
the need for a more robust persistance mechanism will greatly become evident. 
MySQL 5.0.2 is used for our database and most of the recent 
MySQL connector APIs will work with this example.

Almost like Struts, a lot of the hibernate settings 
are defined in a hibernate configuration file, 'hibernate.cfg.xml' 
and your hibernate mapping file, 'Botlist.hbm.xml'. 
Normally the most important settings for your application 
include what database dialect you are using; we are using MySQL 
and the definition of your hibernate POJO beans. 
The simple bean contains an almost one-to-one mapping between 
your database fields and the Java members, accompanied by 
the appropriate getters and setters.

% Basic biology concepts
%%--------------------------------------
%% Scala and Lift
%%--------------------------------------
\section{Standard Java Libraries}

Scala is a JVM language.

Our web-application would not be complete without a clear approach 
for persisting the link data. So we have used the Hibernate ORM 
(object relational mapping) library do the backend persistance work for us. 
It is not really necessary to use Hibernate for such a simple 
application, but as your enterprise application grows, 
the need for a more robust persistance mechanism will greatly become evident. 
MySQL 5.0.2 is used for our database and most of the recent 
MySQL connector APIs will work with this example.

Almost like Struts, a lot of the hibernate settings 
are defined in a hibernate configuration file, 'hibernate.cfg.xml' 
and your hibernate mapping file, 'Botlist.hbm.xml'. 
Normally the most important settings for your application 
include what database dialect you are using; we are using MySQL 
and the definition of your hibernate POJO beans. 
The simple bean contains an almost one-to-one mapping between 
your database fields and the Java members, accompanied by 
the appropriate getters and setters.

% History of AI and ALife other key ai evolution people
%%--------------------------------------
%% Scala and Lift
%%--------------------------------------
\section{Standard Java Libraries}

Scala is a JVM language.

Our web-application would not be complete without a clear approach 
for persisting the link data. So we have used the Hibernate ORM 
(object relational mapping) library do the backend persistance work for us. 
It is not really necessary to use Hibernate for such a simple 
application, but as your enterprise application grows, 
the need for a more robust persistance mechanism will greatly become evident. 
MySQL 5.0.2 is used for our database and most of the recent 
MySQL connector APIs will work with this example.

Almost like Struts, a lot of the hibernate settings 
are defined in a hibernate configuration file, 'hibernate.cfg.xml' 
and your hibernate mapping file, 'Botlist.hbm.xml'. 
Normally the most important settings for your application 
include what database dialect you are using; we are using MySQL 
and the definition of your hibernate POJO beans. 
The simple bean contains an almost one-to-one mapping between 
your database fields and the Java members, accompanied by 
the appropriate getters and setters.

% ALife, AI
%%--------------------------------------
%% Scala and Lift
%%--------------------------------------
\section{Standard Java Libraries}

Scala is a JVM language.

Our web-application would not be complete without a clear approach 
for persisting the link data. So we have used the Hibernate ORM 
(object relational mapping) library do the backend persistance work for us. 
It is not really necessary to use Hibernate for such a simple 
application, but as your enterprise application grows, 
the need for a more robust persistance mechanism will greatly become evident. 
MySQL 5.0.2 is used for our database and most of the recent 
MySQL connector APIs will work with this example.

Almost like Struts, a lot of the hibernate settings 
are defined in a hibernate configuration file, 'hibernate.cfg.xml' 
and your hibernate mapping file, 'Botlist.hbm.xml'. 
Normally the most important settings for your application 
include what database dialect you are using; we are using MySQL 
and the definition of your hibernate POJO beans. 
The simple bean contains an almost one-to-one mapping between 
your database fields and the Java members, accompanied by 
the appropriate getters and setters.

% Our expirement
%%--------------------------------------
%% Scala and Lift
%%--------------------------------------
\section{Standard Java Libraries}

Scala is a JVM language.

Our web-application would not be complete without a clear approach 
for persisting the link data. So we have used the Hibernate ORM 
(object relational mapping) library do the backend persistance work for us. 
It is not really necessary to use Hibernate for such a simple 
application, but as your enterprise application grows, 
the need for a more robust persistance mechanism will greatly become evident. 
MySQL 5.0.2 is used for our database and most of the recent 
MySQL connector APIs will work with this example.

Almost like Struts, a lot of the hibernate settings 
are defined in a hibernate configuration file, 'hibernate.cfg.xml' 
and your hibernate mapping file, 'Botlist.hbm.xml'. 
Normally the most important settings for your application 
include what database dialect you are using; we are using MySQL 
and the definition of your hibernate POJO beans. 
The simple bean contains an almost one-to-one mapping between 
your database fields and the Java members, accompanied by 
the appropriate getters and setters.

% Appendix, running the simulation.
% Configuration, building, etc
%%--------------------------------------
%% Scala and Lift
%%--------------------------------------
\section{Standard Java Libraries}

Scala is a JVM language.

Our web-application would not be complete without a clear approach 
for persisting the link data. So we have used the Hibernate ORM 
(object relational mapping) library do the backend persistance work for us. 
It is not really necessary to use Hibernate for such a simple 
application, but as your enterprise application grows, 
the need for a more robust persistance mechanism will greatly become evident. 
MySQL 5.0.2 is used for our database and most of the recent 
MySQL connector APIs will work with this example.

Almost like Struts, a lot of the hibernate settings 
are defined in a hibernate configuration file, 'hibernate.cfg.xml' 
and your hibernate mapping file, 'Botlist.hbm.xml'. 
Normally the most important settings for your application 
include what database dialect you are using; we are using MySQL 
and the definition of your hibernate POJO beans. 
The simple bean contains an almost one-to-one mapping between 
your database fields and the Java members, accompanied by 
the appropriate getters and setters.

%%--------------------------------------
%% Scala and Lift
%%--------------------------------------
\chapter{Standard Java Libraries}

Scala is a JVM language.

Our web-application would not be complete without a clear approach 
for persisting the link data. So we have used the Hibernate ORM 
(object relational mapping) library do the backend persistance work for us. 
It is not really necessary to use Hibernate for such a simple 
application, but as your enterprise application grows, 
the need for a more robust persistance mechanism will greatly become evident. 
MySQL 5.0.2 is used for our database and most of the recent 
MySQL connector APIs will work with this example.

Almost like Struts, a lot of the hibernate settings 
are defined in a hibernate configuration file, 'hibernate.cfg.xml' 
and your hibernate mapping file, 'Botlist.hbm.xml'. 
Normally the most important settings for your application 
include what database dialect you are using; we are using MySQL 
and the definition of your hibernate POJO beans. 
The simple bean contains an almost one-to-one mapping between 
your database fields and the Java members, accompanied by 
the appropriate getters and setters.

% Figure
\begin{figure}
\centerline{\includegraphics{moreScreenShotThruDemo}}
\caption{Code before preprocessing.}
\label{fig:one}
\end{figure}


\section{MMSN Protocol}

\subsection{Frequency Assignment}

We propose a suboptimal distribution to be used by each node, which is
easy to compute and does not depend on the number of competing
nodes. A natural candidate is an increasing geometric sequence, in
which

where $t=0,{\ldots}\,,T$, and $b$ is a number greater than $1$.

In our algorithm, we use the suboptimal approach for simplicity and
generality. We need to make the distribution of the selected back-off
time slice at each node conform to what is shown in Equation
(\ref{eqn:01}). It is implemented as follows: First, a random
variable $\alpha$ with a uniform distribution within the interval
$(0, 1)$ is generated on each node, then time slice $i$ is selected
according to the following equation:

So protocols [\citeNP{Bahl-02,Culler-01,Zhou-06,Adya-01,Culler-01};
\citeNP{Tzamaloukas-01}; \citeNP{Akyildiz-01}] that use RTS/CTS
controls\footnote{RTS/CTS controls are required to be implemented by
802.11-compliant devices. They can be used as an optional mechanism
to avoid Hidden Terminal Problems in the 802.11 standard and
protocols based on those similar to \citeN{Akyildiz-01} and
\citeN{Adya-01}.} for frequency negotiation and reservation are not
suitable for WSN applications, even though they exhibit good
performance in general wireless ad hoc
networks.

% Head 3
\subsubsection{Exclusive Frequency Assignment}

In exclusive frequency assignment, nodes first exchange their IDs
among two communication hops so that each node knows its two-hop
neighbors' IDs. 

% Head 4
\paragraph{Eavesdropping}

Even though the even selection scheme leads to even sharing of
available frequencies among any two-hop neighborhood, it involves a
number of two-hop broadcasts. To 
\subsection{Basic Notations}

As Algorithm~\ref{alg:one} states, for each frequency
number, each node calculates a random number (${\textit{Rnd}}_{\alpha}$) for
itself and a random number (${\textit{Rnd}}_{\beta}$) for each of its two-hop
neighbors with the same pseudorandom number generator.

Bus masters are divided into two disjoint sets, $\mathcal{M}_{RT}$
and $\mathcal{M}_{NRT}$.
% description
\begin{description}
\item[RT Masters]
$\mathcal{M}_{RT}=\{ \vec{m}_{1},\dots,\vec{m}_{n}\}$ denotes the
$n$ RT masters issuing real-time constrained requests. To model the
current request issued by an $\vec{m}_{i}$ in $\mathcal{M}_{RT}$,
three parameters---the recurrence time $(r_i)$, the service cycle
$(c_i)$, and the relative deadline $(d_i)$---are used, with their
relationships.
\item[NRT Masters]
$\mathcal{M}_{NRT}=\{ \vec{m}_{n+1},\dots,\vec{m}_{n+m}\}$ is a set
of $m$ masters issuing nonreal-time constrained requests. In our
model, each $\vec{m}_{j}$ in $\mathcal{M}_{NRT}$ needs only one
parameter, the service cycle, to model the current request it
issues.
\end{description}

Here, a question may arise, since each node has a global ID. Why
don't we just map nodes' IDs within two hops into a group of
frequency numbers and assign those numbers to all nodes within two
hops?

\section{Simulator}
\label{sec:sim}

If the model checker requests successors of a state which are not
created yet, the state space uses the simulator to create the
successors on-the-fly.

\subsection{Problem Formulation}

The objective of variable coalescence-based offset assignment is to find
both the coalescence scheme and the MWPC on the coalesced graph. We start
with a few definitions and lemmas for variable coalescence.

% Enunciations
\begin{definition}[Coalesced Node (C-Node)]A C-node is a set of
live ranges (webs) in the AG or IG that are coalesced. Nodes within the same
C-node cannot interfere with each other on the IG. Before any coalescing is
done, each live range is a C-node by itself.
\end{definition}

\begin{definition}[C-AG (Coalesced Access Graph)]The C-AG is the access
graph after node coalescence, which is composed of all C-nodes and C-edges.
\end{definition}

\begin{lemma}
The C-MWPC problem is NP-complete.
\end{lemma}
\begin{proof} C-MWPC can be easily reduced to the MWPC problem assuming a
coalescence graph without any edge or a fully connected interference graph.
Therefore, each C-node is an uncoalesced live range after value separation
and C-PC is equivalent to PC. A fully connected interference graph is made
possible when all live ranges interfere with each other. Thus, the C-MWPC
problem is NP-complete.
\end{proof}

\begin{lemma}[Lemma Subhead]The solution to the C-MWPC problem is no
worse than the solution to the MWPC.
\end{lemma}
\begin{proof}
Simply, any solution to the MWPC is also a solution to the
C-MWPC. But some solutions to C-MWPC may not apply to the MWPC (if any
coalescing were made).
\end{proof}

\section{Performance Evaluation}

During all the experiments, the Geographic Forwarding (GF)
\cite{Akyildiz-01} routing protocol is used. GF exploits geographic
information of nodes and conducts local data-forwarding to achieve
end-to-end routing. Our simulation is
configured according to the settings in
Table~\ref{tab:one}. Each run lasts for 2 minutes and
repeated 100 times. For each data value we present in the results,
we also give its 90\% confidence interval.

\section{Conclusions}

In this article, we develop the first multifrequency MAC protocol for
WSN applications in which each device adopts a
single radio transceiver. The different MAC design requirements for
WSNs and general wireless ad-hoc networks are
compared, and a complete WSN multifrequency MAC design (MMSN) is
put forth. During the MMSN design, we analyze and evaluate different
choices for frequency assignments and also discuss the nonuniform
back-off algorithms for the slotted media access design.

% Appendix
\appendix
\section*{APPENDIX}
\setcounter{section}{1}
In this appendix, we measure
the channel switching time of Micaz \cite{CROSSBOW} sensor devices.
In our experiments, one mote alternatingly switches between Channels
11 and 12. Every time after the node switches to a channel, it sends
out a packet immediately and then changes to a new channel as soon
as the transmission is finished. We measure the
number of packets the test mote can send in 10 seconds, denoted as
$N_{1}$. In contrast, we also measure the same value of the test
mote without switching channels, denoted as $N_{2}$. We calculate
the channel-switching time $s$ as
\begin{eqnarray}%
s=\frac{10}{N_{1}}-\frac{10}{N_{2}}. \nonumber
\end{eqnarray}%
By repeating the experiments 100 times, we get the average
channel-switching time of Micaz motes: 24.3$\mu$s.

\appendixhead{ZHOU}

% Acknowledgments
\begin{acks}
The authors would like to thank Dr. Maura Turolla of Telecom
Italia for providing specifications about the application scenario.
\end{acks}

% Bibliography
\bibliographystyle{acmsmall}
\bibliography{acmsmall-sam}

% History dates
\received{March 2011}{March 2011}{March 2011}

% Electronic Appendix
\elecappendix

\medskip

\section{This is an example of Appendix section head}

By repeating experiments 100 times, we get the average
channel-switching time of Micaz motes: 24.3 $\mu$s. We then conduct
the same experiments with different Micaz motes, as well as
experiments with the transmitter switching from Channel 11 to other
channels. In both scenarios, the channel-switching time does not have
obvious changes. (In our experiments, all values are in the range of
23.6 $\mu$s to 24.9 $\mu$s.)

\section{Appendix section head}

The primary consumer of energy in WSNs is idle listening. The key to
reduce idle listening is executing low duty-cycle on nodes. Two
primary approaches are considered in controlling duty-cycles in the
MAC layer.

\end{document}

